% -*- root: ../distributed_hosting_whitepaper.tex -*-

This part gives a motivation and a general overview about the
\textit{Distributed Hosting Engine}. It introduces some concepts that will be
explained in more detail in the next part. If you are looking for a detailed specification you can directly go to Part \ref{part:specifications}.

\section{Hosting Paradigm Shift}\index{Hosting Paradigm Shift}

This whitepaper specifies a new protocol and blockchain that has
the potential to change the current hosting paradigm. Instead of retrieving
pages from a specific location, they are hosted directly on end-user
devices. The distributed hosting algorithm and protocol, specified here, takes
care of the optimal location for each page. By taking into account different
metrics, like response times, availability and/or relevance, the perfect place
for each page will be found after some time. This not only increases the
performance for single page accesses, but reduced also network load and hence
increases the overall network health.

Additional to the increased performance, the underlying blockchain assures the
correctness and validity of each page, also, or especially, if hosted by a
random node in the network. This validation mechanism is then used to create a
reputation system, allowing blocking of malicious nodes that have a reputation
score that is lower than a threshold value.

This mechanism, together with the page distribution algorithm, makes the
network self-managing, self-healing and robust against changes.

By design the \textit{Distributed Hosting Engine} is censorship resistance,
but only under the assumption that there are enough nodes that are willing to
host the page. This moves responsibility what could be hosted from a central
authority to the collective.

From a commercial point of view the here proposed approach allows the creation
of an interoperable hosting ecosystem. Each hosting provider acts as clone and
has no direct access to the page, but can support the user with additional
(paid or free) services and hosting space. By design a switch from one
hosting provider to another one does not affect the hosting of the page, nor
the page itself. Technical there is not even a migration happening, only a
switch from one page creation tool to another one.

\section{Use Cases}\index{Use Cases}

\textbf{TODO:} Explains here different use cases based on user stories of Jane
and John Doe.
